%\documentclass[12pt,letterpaper]{article}
\documentclass[12pt]{amsart}

\usepackage[left=1in, right=1in, top=.5in, bottom=.5in]{geometry}
\usepackage{latexsym,amssymb,amsmath,amsthm,amsopn,verbatim, mathpazo, graphicx, lmodern}
%\usepackage{latexsym,amssymb,amsmath,amsthm,amsopn,verbatim, mathpazo, graphicx}
\newcommand{\vs}{\vskip.5cm}
\newcommand{\hs}{\hskip1cm}
\newcommand{\ds}{\displaystyle}
\setlength{\parindent}{0pt}


\usepackage[T1]{fontenc}
\usepackage{libertine}
\renewcommand*\familydefault{\sfdefault}  %% Only if the base font of the document is to be sans serif


\newtheorem{theorem}{Theorem}[section]
\newtheorem{corollary}{Corollary}[theorem]
\newtheorem{lemma}[theorem]{Lemma}
\newtheorem{definition}[theorem]{Definition}
\newtheorem{example}[theorem]{Example}

\newcommand\indep{\protect\mathpalette{\protect\independenT}{\perp}}
\def\independenT#1#2{\mathrel{\rlap{$#1#2$}\mkern2mu{#1#2}}}

\begin{document}


%----------------------------------------------------------------------------
\setcounter{section}{2}
Day 4 BSTA 511/611
{\huge  
\section*{Chapter 2: Probability (Part 1)}
}

%----------------------------------------------------------------------------


%----------------------------------------------------------------------------
{\large 
%----------------------------------------------------------------------------


%----------------------------------------------------------------------------
%\begin{example}  \textbf{text.} \newline
%
%\begin{itemize}
%\item 
%
%\textbf{Solution:} \newline
%\vspace{2cm}
%
%\item 
%
%\textbf{Solution:} \newline
%
%\vspace{2cm}
%
%\item 
%
%\textbf{Solution:} \newline
%
%
%\vspace{2cm}
%
%
%\item 
%
%\textbf{Solution:} \newline
%
%\end{itemize}
%\end{example} 
%\newpage


%----------------------------------------------------------------------------
\begin{example}  \textbf{Rolling fair 6-sided dice} \newline

\begin{enumerate} 

\item Suppose you roll a fair 6-sided die. 

\begin{enumerate} 
\item What are all the possible outcomes?

%\textbf{Solution:} \newline
\vspace{1.5cm}


\item What is the probability that you roll a 4?

%\textbf{Solution:} \newline
\vspace{1.5cm}

\item What is the probability that you roll an even number?

%\textbf{Solution:} \newline

\vspace{1.5cm}


\item What is the probability that you roll an even number and a 2?

%\textbf{Solution:} \newline

\vspace{1.5cm}


\item What is the probability that you roll a 2 and a 5?

%\textbf{Solution:} \newline

\vspace{1.5cm}


\item What is the probability that you did not roll a 3?

%\textbf{Solution:} \newline

\vspace{1.5cm}


\end{enumerate}

\item Suppose you roll \textbf{two} fair 6-sided dice. 

\begin{enumerate} 
\item What is the probability that you roll a 2 with the first die and a 5 with the second die?

%\textbf{Solution:} \newline
%$$\frac16 \cdot \frac16 = \frac{1}{36}$$
\vspace{1cm}

\item What is the probability that the sum of the two dice is 10?

%\textbf{Solution:} \newline

%$$\mathbb{P}((4,6), (5,5), (6,4)) = 3/36$$
%\vspace{2cm}


\end{enumerate}

\end{enumerate}
\end{example} 


\newpage
%----------------------------------------------------------------------------
%\textbf{Probability Rules}

\subsection{What is a probability?}

%update
\begin{definition} The \textbf{probability} of an outcome is the proportion of times the outcome would occur if the random phenomenon could be observed an infinite number of times.
\end{definition}

\vspace{1cm}
\textbf{Law of Large Numbers}\newline
 As more observations are collected, the proportion of occurrences with a particular outcome converges to the probability $p$ of that outcome.

\vspace{7cm}
\subsection{Probabilities of Equally Likely Outcomes}


\newpage
%----------------------------------------------------------------------------
%\textbf{Probability Rules}

\subsection{Probability Definitions and Rules}
%update numbering

\begin{enumerate}

\item[1.] The \textbf{sample space} $S$ is the set of all possible outcomes.
\vspace{1cm}

\item[2.] \textbf{Events} are sets or collections of outcomes.
\vspace{1cm}

\item[3.] \textbf{Disjoint} or \textbf{mutually exclusive} events\newline
Events $A$ and $B$ are disjoint (or, mutually exclusive) if
%$$\mathbb{P}(A\ \mathrm{and}\  B) = 0$$
\vspace{3cm}

\item[4.] \textbf{Addition Rule for Disjoint Events}\newline
If $A$ and $B$ are disjoint events,  then the probability that at least one of them will occur is 
%	$$\mathbb{P}(A\ \mathrm{or}\  B)=\mathbb{P}(A) + \mathbb{P}(B)$$
\vspace{3cm}


\item[5.] \textbf{General Addition Rule}\newline
If $A$ and $B$ are any two events, disjoint or not, then the probability that at least one of them will occur is
%	$$\mathbb{P}(A\ \mathrm{or}\  B)=\mathbb{P}(A) + \mathbb{P}(B) - \mathbb{P}(A\ \mathrm{and}\  B)$$
\vspace{3cm}


\item[6.] \textbf{Complement}\newline
The complement of event $A$ is denoted $A^C$, and $A^C$ represents all outcomes not in $A$.
%	$$\mathbb{P}(A) + \mathbb{P}(A^C) = 1$$
\vspace{1cm}


\end{enumerate}

\newpage
%----------------------------------------------------------------------------
\begin{example}  \textbf{Diabetes and hypertension.} \newline
 Diabetes and hypertension are two of the most common diseases in Western, industrialized nations. In the United States, approximately 9\% of the population have diabetes, while about 30\% of adults have high blood pressure. The two diseases frequently occur together: an estimated 6\% of the population have both diabetes and hypertension. \newline
Let D represent the event of having diabetes, and H the event of having hypertension. 

\begin{enumerate}

\item Are the events having hypertension and having diabetes mutually exclusive?

%\textbf{Solution:} \newline

\vspace{3cm}

\item Draw a Venn diagram summarizing the variables and their associated probabilities.

%\textbf{Solution:} \newline
\vspace{5cm}

\item Calculate the probability of having diabetes or hypertension.

%\textbf{Solution:} \newline
%$$P(D\ or\ H) = .09 + .3 - .06 =  0.33$$
\vspace{4cm}

\item What percent of Americans have neither hypertension nor diabetes?

%\textbf{Solution:} \newline
%$$1 - P(D\ or\ H) = 1-   0.33 = .67 = 67\%$$
%\vspace{2cm}


\end{enumerate}
\end{example} 





\newpage
%----------------------------------------------------------------------------
\begin{example}Pulling cards from a standard deck of cards

\begin{enumerate}
\item Supposed you randomly pull 5 cards one after another from a standard deck of cards, 

\begin{enumerate}
\item \textbf{replacing} a card before pulling another one. What is the probability of pulling the sequence of $\diamondsuit, \diamondsuit, \spadesuit, \heartsuit, \spadesuit$?

%\textbf{Solution:} \newline
%$$\mathbb{P}(\diamondsuit, \diamondsuit, \spadesuit, \heartsuit, \spadesuit) = 
%\frac{13}{52}\cdot\frac{13}{52}\cdot\frac{13}{52}\cdot\frac{13}{52}\cdot\frac{13}{52} = \Big(\frac{13}{52}\Big)^5$$

\vspace{4cm}

\item \textbf{without replacing} any of the cards. What is the probability of pulling the sequence of $\diamondsuit, \diamondsuit, \spadesuit, \heartsuit, \spadesuit$?

%\textbf{Solution:} \newline
%$$\mathbb{P}(\diamondsuit, \diamondsuit, \spadesuit, \heartsuit, \spadesuit) = 
%\frac{13}{52} \cdot \frac{12}{51} \cdot \frac{13}{50} \cdot\frac{13}{49} \cdot\frac{12}{48}$$

\vspace{4cm}
\end{enumerate}

\item Supposed you pull a red card from the deck. What is the probability that it is a heart?

%\textbf{Solution:} \newline
%$$\mathbb{P}(\heartsuit | red) = \frac{\mathbb{P}(\heartsuit\ and\ red)}{\mathbb{P}(red)} = \frac{1/4}{1/2} = \frac{1}{2}$$


\end{enumerate}

\end{example}





\newpage
%----------------------------------------------------------------------------
%\textbf{Probability Rules}

\subsection{More Probability Rules}

\begin{enumerate}
\item[1.] \textbf{General Multiplication Rule}\newline 
For events $A$ and $B$, 
%	$$\mathbb{P}(A\ \mathrm{and}\  B)=\mathbb{P}(A)\cdot\mathbb{P}(B|A)$$
$$\mathbb{P}(A\ \mathrm{and}\  B)=$$
\vspace{2cm}

\item[2.] \textbf{Conditional Probability Definition}\newline 
The conditional probability of an event A given an event or condition B is

%	$$\mathbb{P}(A|B)=\frac{\mathbb{P}(A\cap B)}{\mathbb{P}(B)}$$
		$$\mathbb{P}(A|B)=$$
\vspace{2cm}

\item[3.] If $A$ and $B$ are \textbf{independent} events, then

\begin{enumerate}
\item 
%$$\mathbb{P}(A|B) = \mathbb{P}(A)$$
$$\mathbb{P}(A|B) = $$

\vspace{2cm}
\item $$\mathbb{P}(A\ \mathrm{and}\  B)=$$
\end{enumerate}

	
\vspace{2cm}

\item[4.] \textbf{Sum of Conditional Probabilities}\newline 
$\mathbb{P}(A|B)$ is a probability, meaning that it satisfies the usual probability rules. In particular, 
%	$$\mathbb{P}(A|B) + \mathbb{P}(A^C|B) = 1$$	
	$$\mathbb{P}(A|B) + \mathbb{P}(A^C|B) = $$	

\end{enumerate}



\newpage
%----------------------------------------------------------------------------

%\subsection{Independence}

%----------------------------------------------------------------------------
\begin{example}   \textbf{Diabetes and hypertension revisited.} \newline
Diabetes and hypertension are two of the most common diseases in Western, industrialized nations. In the United States, approximately 9\% of the population have diabetes, while about 30\% of adults have high blood pressure. The two diseases frequently occur together: an estimated 6\% of the population have both diabetes and hypertension. \newline
Let D represent the event of having diabetes, and H the event of having hypertension. 
Is the event of someone being hypertensive independent of the event that someone has diabetes?

%\textbf{Solution:} \newline

%No. 
%$$P(D\ and\ H) = .06 \neq  P(D)\cdot P(H) = .09 \cdot .3 =  .027$$


\end{example} 


\newpage
%----------------------------------------------------------------------------
\begin{example}\textbf{Seat belts} \newline
Seat belt use is the most effective way to save lives and reduce injuries in motor vehicle crashes. In a 2014 survey, respondents were asked, "How often do you use seat belts when you drive or ride in a car?". The following table shows the distribution of seat belt usage by sex.
\vspace{.2cm}
\begin{center}
		\begin{tabular}{rrrrrrrr}
			& &  \multicolumn{5}{c}{\textit{Seat Belt Usage}} &  \\ 
			\cline{3-7}
			&       & Always & Nearly always & Sometimes    & Seldom     & Never  & Total \\ 
			\cline{2-8}
%			\multirow{2}{*}{\textit{Sex}}    & Male    & 146,018   & 19,492    & 7,614   &  3,145  & 4,719 & 180,988 \\ 
			\textit{Sex}    & Male    & 146,018   & 19,492    & 7,614   &  3,145  & 4,719 & 180,988 \\ 
			& Female   & 229,246    & 16,695    & 5,549    & 1,815  & 2,675 &  255,980 \\ 
			\cline{2-8}
			& Total & 375,264    & 36,187    & 13,163    & 4,960   & 7,394  &  436,968
		\end{tabular}
\end{center}
\vspace{.2cm}

\begin{enumerate}
\item Calculate the marginal probability that a randomly chosen individual always wears seatbelts.

%375, 264/ 436, 968 = 0.859
\vspace{3cm}

\item What is the probability that a randomly chosen female always wears seatbelts?

%229, 246/ 255, 980 = 0.896

\vspace{3cm}

\item What is the probability of a randomly chosen individual always wearing seatbelts, given that they are female? 

same as previous
\vspace{3cm}

\item What is the probability of a randomly chosen individual being female and always wearing seatbelts? 

%229, 246/ 436, 968 = 

\vspace{3cm}

%\end{enumerate}


%\end{example}




\newpage
%----------------------------------------------------------------------------
%\begin{example}
\textbf{Seat belts cont'd} \newline

\vspace{.2cm}
\begin{center}
		\begin{tabular}{rrrrrrrr}
			& &  \multicolumn{5}{c}{\textit{Seat Belt Usage}} &  \\ 
			\cline{3-7}
			&       & Always & Nearly always & Sometimes    & Seldom     & Never  & Total \\ 
			\cline{2-8}
%			\multirow{2}{*}{\textit{Sex}}    & Male    & 146,018   & 19,492    & 7,614   &  3,145  & 4,719 & 180,988 \\ 
			\textit{Sex}    & Male    & 146,018   & 19,492    & 7,614   &  3,145  & 4,719 & 180,988 \\ 
			& Female   & 229,246    & 16,695    & 5,549    & 1,815  & 2,675 &  255,980 \\ 
			\cline{2-8}
			& Total & 375,264    & 36,187    & 13,163    & 4,960   & 7,394  &  436,968
		\end{tabular}
\end{center}
\vspace{.2cm}

%\begin{enumerate}


\item What is the probability of a randomly chosen individual always wearing seatbelts, given that they are male?

%146, 018/ 180, 988 = 0.807
\vspace{3cm}

\item Calculate the probability that an individual who never wears seatbelts is male.

%4, 719/ 7, 394 = 0.638
\vspace{4cm}

\item Does gender seem independent of seat belt usage?

%No, because the answers to (3) and (5) are not equal. If gender and seat belt usage were independent, then among males and females, there would be the same proportion of people who always wear seat belts.
\end{enumerate}


\end{example}




%----------------------------------------------------------------------------
}  % end large font
%----------------------------------------------------------------------------



\end{document}

